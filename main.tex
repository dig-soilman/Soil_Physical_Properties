
\documentclass[a5paper]{book}

\usepackage[utf8]{inputenc}

\usepackage[a5paper, parfill, inner=1.5cm, outer=1.5cm, top=1.75cm, bottom=1.5cm]{geometry} % this will be half an A4

\usepackage{amsmath} % provides miscellaneous enhancements for improving the information structure and printed output of documents containing mathematical formulas 

\usepackage{amsfonts,amssymb} % amssymb provides an extended symbol collection and internally loads amsfonts

\usepackage{graphicx} % show images

\usepackage{multirow} % for multirow in tables

\usepackage{array}

\graphicspath{ {images/} } % Path relative to the .tex file containing the \includegraphics command

% begin itemize and enumerate formatting
\usepackage{enumitem} % Control layout of itemize, enumerate, description
\setlist{nolistsep} % removes space between list and enumerate items
% end itemize and enumerate formatting

% begin paragraph formatting
\usepackage{parskip}
\setlength{\parindent}{0pt} % Default is 15pt thus no first line indent
%\setlength{\parskip}{2mm plus1mm minus1mm} % adds space between paragraphs
\setlength{\parskip}{\baselineskip} % adds space between paragraphs
% end paragraph formatting


%+Title
  \title{Soils: Physical Properties}
  \author{Donald G. McGahan\\ Professor Tarleton State University}
  \date{May 6, 2022}
%-Title

\begin{document}

\maketitle

Descriptions of the physical properties of soil include size separates, organization of size separates, density, porosity, consistency, color, and temperature.
    
\section{Organic Soil}
\label{organic_soil}
    
Soil solids can be mineral or organic in nature. A soil that consists of primarily accumulated plant material of various degrees of decomposition are separated from soils derived from dominantly mineral materials. The use of the term \textbf{organic soil} in this text is that defined by the U.S. Department of Agriculture. To be considered an organic soil:
    
\begin{enumerate}
    \item Soil that is never saturated with water for more than a few days must contain more than 20 percent organic carbon.
    \item If the soil is saturated for periods longer than a few days, it is organic if it contains the following:
    \begin{itemize}
        \item At least 12 percent organic carbon if the soil has not clay,
        \item At least 18 percent organic carbon if the soil has 60 percent or more clay, or
        \item If it contains a proportional amount of organic carbon for intermediate amounts of clay.
    \end{itemize}
\end{enumerate}
    
The percentages are determined on a mass basis. A conversion factor of 1.72 is commonly used to convert organic carbon to organic matter. This conversion factor assumes organic matter contains 58 \% organic carbon. However this can vary with the type of organic matter, soil type and soil depth. Conversion factors can be as high as 2.50, especially for subsoils.
    
Most soils have some organic carbon but are below the levels necessary to attain the organic soil name. The inorganic or mineral fraction of the soil are measured and classified based on size.
    
\section{Classifying Mineral Size Separates}
    
Separates are the individual particles that together with organics, salts, and evaporates comprise a soil. We size them by placing them together in groupings that have a range in size with an upper and lower effective spherical diameter. The choice for the range in size is dictated by properties that one ‘size range’ exhibits, that is different than other size ranges. Often these can be grouped for particular purposes. The groupings once formalized are termed ‘classes.’ The term fraction is used to connote part of a whole.
    
First, the standard is determination based on only the mineral fraction. Organics, salts, and evaporites are not included as part of the whole.
    
Second, we group the separates of the “whole” soil into two classes, the coarse fraction (CF) and the fine-earth fraction. The coarse fraction consists of the particles greater than 2~mm in diameter and the fine-earth fraction consist of particles equal to and less than 2~mm.
    
Both the coarse fraction and fine-earth fraction are further divided, again, on the basis of size. The fine-earth fraction is divided first into three size separates: sand, silt, and clay.
    
Coarse Fraction are separates greater in diameter than 2~mm.
    
Fine-Earth Fraction are separates less than or equal to 2~mm.
    
Coarse fragments, or the coarse fraction, are important to consider because they occupy space in the soil, but contribute little or no porosity (water storage) and chemical reactivity (nutrient storage). Thus, coarse fragments reduce the soil volume available to hold water and nutrients. Coarse fragments may also make cultivation difficult.

\subsection{Soil Texture}
    
Soil texture refers to the proportion groupings of size separates in the inorganic soil fine earth fraction (\textless 2-mm in diameter). There are generally three groupings with further divisions of the three to achieve more refined interpretive relevance. Texture can generally not be changed, except at great expense (example: adding sand to your garden). The three groupings of the fine earth fraction are sand, silt, and clay.
    
The sands are the separates between 0.05 to 2~mm.

The sand size separates are further classed into five size classes. \emph{Very Course Sand} separates are from 2~mm to 1~mm diameter. \emph{Course Sand} separates are from 1~mm to 0.5~mm diameter. \emph{Medium Sand} separates is from 0.5~mm to 0.25~mm diameter. \emph{Fine Sand} separates is from 0.25~mm to 0.10~mm diameter. \emph{Very Fine Sand} separates is from 0.01 to 0.05~mm diameter.
    
Silt are separates are between the diameters of 0.002 – 0.05 mm or 2 – 50 \textmu m, where \textmu\, is micro.

Clay are separates with diameters of \textless\,0.002 mm or \textless\,2 \textmu m.

\subsection{Textural Triangle}
    
The textural triangle relates the relationship between the sand, silt, and clay mass proportions and helps place the texture into a Textural Class. There are twelve soil textural classes.
    
Sand, Loamy Sand, Sandy Loam, Sandy Clay Loam, Silt, Silt Loam, Silty Clay Loam, Loam, Clay Loam, Clay, Silty Clay, and Sandy Clay.
    
It is important to note that the sand, silt, and clay mass proportions represented as a percent of the fine earth fraction add up to 100 percent. Organic matter and course fraction is not included when determining the textural class.
    
Sand\% + Silt\% + Clay\% = 100\%
    
\begin{figure*}
    \centering
    \includegraphics{images/TexturalTriangle.jpg}
    \caption{Gibbs Diagram Textural Triangle}
    \label{fig:TexturalTriangle}
\end{figure*}
    
Loam is a soil textural class in which the sand, silt and clay fractions have a similar influence on soil properties.
    
The textural class can be further subdivided. This usually based upon sand distribution e.g. Loamy Very Fine Sand.
    
The textural class name has a coarse fraction modifier added when the coarse fraction exceeds 15\% of the soil horizons volume.
    
The Course Fraction is divided into four (4) classes: \emph{gravel}, \emph{cobble}, \emph{stone}, and \emph{boulder}.
    
The implications of course fraction is that course fraction does not participate in water retention and can have an impact on tillage operations.
    
If two or more coarse fraction sizes exist the entire volume of the course fraction is used for the modifier.
    
The largest size fraction is named, unless a smaller fraction is more than twice the volume of the larger size fraction and then use that smaller fraction modifier.
    
\subsection{Determination of soil texture}

The field method of determining textural class is the "feel method” where sand feels gritty, silt feels smooth or floury, and clay feels sticky. And the subsequently determined proportions are classed and as “apparent texture”.

\begin{figure*}
    \includegraphics[width=0.9\columnwidth]{images/TextureFlowChart.png}
    \caption{Texture-by-Feel decision flow chart to determine Textural Class}
    \label{fig:TextureByFeelFlowChart}
\end{figure*}

Laboratory procedures to determine the actual textural class begins by separation of the coarse fraction from the fine earth fraction via a 2-mm sieve. The coarse fraction volume is determined utilizing Archimedes principle of liquid displacement. The coarse fraction is reported as volume whereas the fine earth fractions are reported as mass.

Pretreatments to the 2~mm and smaller mineral fraction enhance separation of the size separates. Coarser organic matter is removed and the humus fraction of organic matter is oxidized with hydrogen peroxide. Salts can be removed by passing water through the soil. Carbonates can be evolved with a weak acid such as glacial acetic acid and mild heat. Iron can be reduced and chelated for removal.

The remaining 2~mm and smaller mineral fraction separates are then dispersed with sodium hexametaphosphate with mixing and the 2000 to 50 \textmu m sands are separated by sieving from the $\leq$ 50 \textmu m separates silt and clay, fraction. Sampling an aliquot of the suspension at a specified depth via a pipette to isolate the $\leq$ 2 \textmu m clay follows dispersion and sedimentation.

Such pipette sampling is considered to be the “platinum standard” for textural analysis.

Sampling time for the plane above where the silt fraction has passed below is determined via a modified form of Stokes’ Law for sedimenting soil suspensions:

\begin{equation}
    t=\frac{18\eta h} {[g\,(\rho_s - \rho_l)X^2]} \\
    \begin{aligned}
        \text{where} \\
        t &= \text{time} \\
        \eta &= \text{fluid viscosity} \\
        h &= \text{-height above plane z; the sampling depth} \\
        g &= \text{acceleration due to gravity} \\
        \rho_s &= \text{particle density solid} \\
        \rho_l &= \text{liquid density} \\
    \end{aligned}
\end{equation}
        
Frequently the density of the size separates are assumed and the particles are assumed to be spherical.

Bouyoucos devised a hydrometer method in an effort to simplify the measurement of sand, silt, and clay for a given soil. The initial methodology required two readings. The initial reading at 40 seconds to differentiate between the mass of soil without sand proves to be quite inaccurate. The modified Bouyoucos forgoes the 40 second reading and calls for sieving the sand out of the suspension. This negates much of the time savings of the original method. Additionally with textural classes sand, sandy loam, and loamy sand the sand might need to be fractionated anyway for proper subclass modification of the sand.

The sands are further fractionated via dry sieving. The amount of each of the sand sub-fractions becomes important for soils with textural class of sand, loamy sand, or sandy loam soils wherein the sand might earn a \emph{coarse}, \emph{fine} or \emph{very fine} modifier. See Soil Survey Manual 2017 for more on sand textures subclasses.

Note that the silt fraction is further separated into fine and coarse. The silt size has particular importance to the mechanical properties of the silt size and is useful for classing for erosion.

In an introductory soils course students are responsible for the size limits of the coarse fraction and fine earth fraction. Within the fine earth fraction introductory soils course students are responsible for the size limits of clay, silt, and sand.

The clay and sand are also subdivided. Memorizing the sand fractions is particularly useful since these sizes often show up in names of soil map units. Soil map units convey information about patterns of spatial variability across the landscape. Additionally, in more advanced courses students are typically asked to be responsible for the size limits of the fine silt and coarse silt, and occasionally even the clay which is fractionated into
    
\begin{itemize}
    \item fine clay $\left( < 0.00008\, mm \right)$ or $\left( < 0.08\, \mu m \right)$
    \item medium clay $\left(0.0002\, \text{to}\, 0.00008\, mm \right)$, or $\left(0.02 \mu m\,\text{to}\,0.08\, \mu m \right)$
    \item coarse clay $\left(0.002\, \mu m\, \text{to}\, 0.0002\, mm\right)$, or $\left(2\, \mu m\,\text{to}\,\mu m \right)$
\end{itemize}
    
The amount of fine clay is particularly important when classifying some soils.
    
\begin{table}[!htbp]
\label{tab:sandfractions}
\centering
\caption{Table of sand fraction names and size limits.}
\begin{tabular}{|l|c|}
\hline
Sand Separate           &  Diameter (mm)\\
\hline \hline
very fine sand          & 0.05 - 0.10\\
fine sand               & 0.10 - 0.25\\
medium sand             & 0.25 - 0.50\\
coarse sand             & 0.50 - 1.00\\
very coarse sand        & 1.00 - 2.00\\
\hline
\end{tabular}
\end{table}
    
\begin{table}[!htbp]
\label{tab:coarsefractionvolume}
\centering
\caption{Table of modifier names with the volume of the coarse fraction.}
\begin{tabular}{|l|c|}
\hline
Volume of Coarse Fraction           &  Modifier Name\\
\hline \hline
0-15\%      & no modifier\\
15-35\%     & gravelly, cobbly, stony, or bouldery\\
35-60\%     & add \textbf{very} to the modifier e.g. very gravelly\\
\textgreater\,60\%      & add \textbf{extremely} to the modifier e.g. very gravelly\\
\hline
\end{tabular}
\end{table}
    
\begin{table}[!htbp]
\label{tab:coarsefractionsizenames}
\centering
\caption{Table of coarse fraction separates size increments.}
\begin{tabular}{|l|c|}
\hline
Volume of Coarse Fraction           &  Modifier Name\\
\hline \hline
Gravel      & 2-75\,mm \\
Cobble      & 75-250\,mm \\
Stone       & 250-600\,mm \\
Boulder     & \textgreater\, 600\,mm \\
\hline
\end{tabular}
\end{table}
    
\section{Soil Structure}
    
\textbf{Structure} is the term used to convey the arrangement of size separates pedologically arranged into larger stable units called \textbf{aggregates}, or \textbf{peds}, in the soil expressed as regularly repeating pattern of permanent cleavage planes in the soil mass. The soil aggregates cleave planes form due to repeated compression and contraction of soil particles which may be related to shrink/swell of clay, freeze-thaw cycles, root penetration through soil animals (earth worm activity), etc. The relative mechanical and water stability of peds are variable and related to the aggregate promoting substances organics, iron oxides, carbonates, clays, and/or silica. Organics can be from decomposed vegetative material and/or root polysaccharids exuded from roots.
    
Soils can be structureless expressing a single grain or cohesive nature. Cohesive structureless soils and structured soil can produce “artificial”–not pedogeneticaly derived–\textbf{clods} when broken up by mechanical means. Clods are different that using the term \textbf{fragment} which is used to describe a piece of a broken ped.
    
\textbf{Nodules} and \textbf{concretions} are two other arrangements encountered that have formed in soils and while both are cemented spherical bodies without obvious crystals the concretions display layering whereas nodules do not.
    
Structure influences water movement, heat transfer, aeration, porosity, and erosion as a secondary influence to the overarching constraints and influences arising due to the soil texture. Unlike soil texture, however, soil structure can be changed by management practices.
    
Soil structure is observed, measured, and recorded in-situ as \textbf{Type}, \textbf{Grade}, and \textbf{Size}.
    
\subsection{Type of Structure}
    
Classing soil structure is first expressed by the form now called type and formally referred to as shape. Eight types exist: granular, angular blocky, sub angular blocky, lenticular, platy, wedge, prismatic, and columnar.

The prismatic and columnar type structure are taller than wide with the prismatic having flat tops and columnar having rounded tops that are commonly described as having "bleached" tops. The bleached tops result from dispersion and removal of pigmenting agents such as iron and manganese oxides and organic matters that so often adorn the outside of peds.

Granular and blocky structural types are roughly equidistant side to sided and top to bottom. Granular are polyhedral with irregular faces that are difficult to imagine a clear nesting faces with their neighbors. Blocky structures planar faces are discernible as nesting with neighboring blocks. Angular describes blocky structure type with sharp angular faces versus the subangular blocky polyhedrals wherein the planar faces lack sharp angles and are subrounded.

Wedge type structure is elliptical and terminate at acute angles, often 30\textdegree\, or 60\textdegree, with slickenside bounding edges. While the wedge structure type does not require the soil be a U.S. Soil Taxonomy Vertisol Order, the Vertisol Order, and Vertic taxonomic subgroups must include wedge type structure.

Lenticular type structure is overlapping, lens-shaped peds that are  generally parallel to the soil surface that are thick at the center and taper toward the edges and are formed by active or relict periglacial frost processes. Most common in soils with moderate to high water-holding capacity in moist conditions.

Platy type structure are flat plate like units that are wider than tall.

\begin{figure*}
    \centering
    \includegraphics[width=0.8\columnwidth]{images/FieldBookVer3_ExamplesSoilStructureTypes1045x1145.png}
    \caption{Types of Soil Structure. From Soil Survey Staff 2012 Field Book for Describing and Sampling Soils version 3.}
    \label{fig:SoilStructureTypes}
\end{figure*}

\subsection{Grade of Structure}
    
Three grades of structure are described.

\begin{itemize}
    \item \textbf{\emph{Weak}}: Peds are barely distinguishable in part of the \emph{moist} soil; only a few distinct peds can be separated from the soil mass.
    \item \textbf{\emph{Moderate}}: Peds are visible in place; many can be handled without breaking.
    \item \textbf{\emph{Strong}}: Most of the soil mass is visible as beds, moist of which can be handled with ease without breaking.
\end{itemize}

Soil structure may exist as a compound structure in which large peds may further fall apart into smaller blocks or smaller peds. Both are described starting with the larger type the term "parting to" and then the smaller type add to the description.
    
\subsection{Size of Structure}
    
Size is conveyed in such a way that the peds narrowest dimension  determines the size class for each type. Size classes for granular, angular blocky, sub-angular blocky, lenticular, wedge, prismatic, and columnar structure types are \emph{very fine}, \emph{fine}, \emph{medium}, \emph{coarse}, and \emph{very coarse}. For the platy structure type thin is substituted for fine and thick is substituted for coarse size class names.

\begin{table}[!htbp]
\label{tab:astructuresizeclass}
\centering
\caption{Structure size classes.}
\begin{tabular}{c c c c c}
\hline
\rule{0pt}{1.75em}Size Class & Code & \multicolumn{3}{c}{Criteria: Structural unit size (mm)} \\[0.75em] \cline{3-5} 
\rule{0pt}{1.75em} && \parbox[c]{2cm}{Granular, Platy\\(Thickness)} & \parbox[c]{2cm}{Columnar, Prismatic, Wedge\\(Diameter)} & \parbox[c]{2cm}{Angular \& Subangular Blocky and Lenticular\\(Diameter)} \\[0.75em]
\rule{0pt}{1.75em}\parbox[c]{2cm}{Very Fine\\(Very Thin)} & \parbox[c]{1cm}{VF\\(VN)} & \textless\,1 & \textless\,10 & \textless\,5 \\[0.75em]
\rule{0pt}{1.75em}\parbox[c]{2cm}{Fine\\(Thin)} & \parbox[c]{1cm}{F\\(TN)} & \textless\,1\,to\,\textless\,2 & 10\,to\,\textless\,20 & 5\,to\,\textless\,10 \\[0.75em]
\rule{0pt}{1.75em}\parbox[c]{2cm}{Coarse\\(Thick)} & \parbox[c]{1cm}{CO\\(TK)} & 5\,to\,\textless\,10 & \textless\,50\,to\,\textless\,100 & \textless\,20\,to\,\textless\,50 \\[0.75em]
\rule{0pt}{1.75em}\parbox[c]{2cm}{Very Coarse\\(Very Thick)} & \parbox[c]{1cm}{VC\\(VK)} & $\geq$\,10 & \textless\,100\,to\,\textless\,500 & $\geq$\,50 \\[0.75em]
\rule{0pt}{1.75em}\rule[-1em]{0pt}{1em}\parbox[c]{2cm}{Extremely Coarse} & \parbox[c]{1cm}{EC\\(\textendash)} & \textendash & $\geq$\,500 & \textendash \\
\hline
\end{tabular}
\end{table}
    
\subsection{Destruction of Aggregates}
    
Increasing sodium tends to speed structure degradation. The sodium interacts with the soil particles promoting dispersion. Dispersed size separates are frequently in clay size class and greatly diminish the movement of water and this limits the removal of salts. Tillage also destroys structure and though it can temporarily fluff up the size separates it tends to promote the decomposition of organic matter and the organic matter, as stated previously, is an aggregating agent that promotes structure stability. Trafficking by animals and machinery also have a negative impact on soil structure.
    

\section{Density}
    
Solids Density $\left(\rho_s\right)$ is the density of the solid soil particles only in units of mass per volume e.g., $\left(\frac{g}{cm^3}\right)$. Often in texts the greek letter rho  is a shorthand symbol for density and s is for solids.

\begin{equation}
    Particle\,Density = \left(\frac{Weight\,of\,Solids}{Volume\,of\,Solids}\right)   
\end{equation}

Expressed in a maths form $\rho_p = {W_s \over V_s}$ where $\rho$ is density, $p$ is particle, $W$ is weight, $V$ is volume, and $s$ is solid.

\section{Soil Color}
    
Most soil minerals are not highly colored (often light gray); color is due to the presence of materials that coat the surfaces of soil minerals. Soil color can tell us a lot about a soil (e.g., drainage condition, organic matter content \& relative degree of chemical weathering); however, it has little affect on the behavior and use of soils.
    
The Munsell Book of Color
    
Example: 10YR 5/4
    
\begin{itemize}
    \item Hue = 10 YR
    \item Value = 5
    \item Chroma = 4
\end{itemize}
    
Hue is the spectral color or dominant wavelength
    
\subsection{Redoximorphic Features}
    
Redoximorphic features were formerly referred to as mottles in older texts but the term mottles are now reserved for geogenic features.
    
Redoximorphic features are spots or patches of different color or shades of color interspersed with the dominant soil matrix color. Their presence indicates alternating periods of good drainage with poor drainage. Examples include bright red or yellow patches in a grayish soil matrix (background).


\end{document}