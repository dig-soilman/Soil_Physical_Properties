% Donald G. McGahan 2022 v. 01
%\documentclass{memoir}
\documentclass{article}
\usepackage[utf8]{inputenc}
\usepackage{array}
%\usepackage{multirow} % for multi-row in tables
\begin{document}
\subsection{Size of Structure}
\label{sizeofstructure}
    
Size is conveyed in such a way that the peds narrowest dimension  determines the size class for each type. Size classes for granular, angular blocky, sub-angular blocky, lenticular, wedge, prismatic, and columnar structure types are \emph{very fine}, \emph{fine}, \emph{medium}, \emph{coarse}, and \emph{very coarse}. For the platy structure type thin is substituted for fine and thick is substituted for coarse size class names.

% dig-soilman: revised but waiting to see if this structure size class table translates to html via pandoc well.

\begin{table}[!htbp]
\label{tab:astructuresizeclass}
\centering
\caption{Structure size classes.}
\begin{tabular}{ m{6em} m{5.5em} m{5.5em} m{5.5em} } \\[0.75em]
Size Class & \multicolumn{3}{c}{Criteria: Structural unit size (mm)} \\[0.75em] 
\cline{2-4} \\
& Granular \mbox{(Diameter)}, Platy \mbox{(Thickness)} & Columnar, Prismatic, Wedge, \mbox{(Diameter)} & Angular \& Subangular Blocky and Lenticular, \mbox{(Diameter)} \\[0.75em]
 \hline\\
{Very Fine, \mbox{(Very Thin)}} & \hfil \textless 1 &  \hfil\textless 10 &  \hfil\textless 5 \\[0.75em]
Fine, \mbox{(Thin)} & \hfil 1 to \textless 2 & \hfil 10 to \textless 20 & \hfil 5 to \textless 10 \\[0.75em]
{Medium, \mbox{(Medium)}} & \hfil 2 to \textless 5 & \hfil 20 to \textless 50 & \hfil 10 to \textless 20 \\[0.75em]
{Coarse, \mbox{(Thick)}} & \hfil 5 to \textless 10 & \hfil 50 to \textless 100 & \hfil \textless 20 to \textless 50 \\[0.75em]
{Very Coarse, \mbox{(Very Thick)}} & \hfil $\geq$ 10 & \hfil 100 to \textless 500 & \hfil $\geq$ 50 \\[0.75em]
{Extremely Coarse} & \hfil \textendash & \hfil $\geq$ 500 & \hfil \textendash \\[0.75em]
\hline
\end{tabular}
\end{table}

% dig-soilman: below is an html rendering of the above table.
%<table id="tab:astructuresizeclass">
%	<caption>Table Structure size classes.</caption>
%<tbody>
%	<tr class="odd">
%		<th rowspan="2" style="text-align: left;">Size Class</th>
%		<th colspan="3" style="text-align: center;">Criteria: Structural unit size (mm)</th>
%	</tr>
%	<tr class="even">
%		<th style="text-align: center;">Granular, (Diameter) Platy (Thickness)</th>
%		<th style="text-align: center;">Columnar, Prismatic, Wedge, (Diameter)</th>
%		<th style="text-align: center;">Angular &amp; Subangular Blocky and Lenticular, (Diameter)</th>
%	</tr>
%	<tr class="odd">
%		<td style="text-align: left;">Very Fine (Very Thin)</td>
%		<td style="text-align: center;">&lt; 1</td>
%		<td style="text-align: center;">&lt; 10</td>
%		<td style="text-align: center;">&lt; 5</td>
%	</tr>
%		<td style="text-align: center;">1 to &lt; 2</td>
%		<td style="text-align: center;">10 to  &lt;  20</td>
%		<td style="text-align: center;">5 to &lt; 10</td>
%	</tr>
%	<tr class="odd">
%		<td style="text-align: left;">Medium (Medium)</td>
%		<td style="text-align: center;">2 to &lt; 5</td>
%		<td style="text-align: center;">20 to &lt; 50</td>
%		<td style="text-align: center;">10 to &lt; 20</td>
%	</tr>
%		<tr class="even">
%		<td style="text-align: left;">Coarse (Thick)</td>
%		<td style="text-align: center;">5 to &lt;  10</td>
%		<td style="text-align: center;">50 to &lt; 100</td>
%		<td style="text-align: center;">20 to &lt; 50</td>
%	</tr>
%	<tr class="odd">
%		<td style="text-align: left;">Very Coarse (Very Thick)</td>
%		<td style="text-align: center;">&ge; 10</td>
%		<td style="text-align: center;">100 to &lt; 500</td>
%		<td style="text-align: center;">&ge; 50</td>
%	</tr>
%		<tr class="even">
%		<td style="text-align: left;">Extreamly Coarse</td>
%		<td style="text-align: center;">&ndash;</td>
%		<td style="text-align: center;">&ge; 500</td>
%		<td style="text-align: center;">&ndash;</td>
%	</tr>
%</tbody>
%</table>


% dig-soilman: the following table looks fine when a PDF is created from pdflatex but does not translate well using pandoc to output to html
\begin{table}[!htbp]
\label{tab:astructuresizeclass}
\centering
\caption{Structure size classes.}
\begin{tabular}{l c c c }
\hline \\[0.15em]
Size Class & \multicolumn{3}{c}{Criteria: Structural unit size (mm)}\\[0.75em] 
\cline{2-4} \\
& \parbox[c]{2cm}{Granular, \mbox{(Diameter)} Platy \mbox{(Thickness)}} & \parbox[c]{2cm}{Columnar, Prismatic, Wedge, \mbox{(Diameter)}} & \parbox[c]{2cm}{Angular \& Subangular Blocky and Lenticular, \mbox{(Diameter)}} \\[0.75em]
 \hline\\[0.75em]
{Very Fine, \mbox{(Very Thin)}} & \textless 1 & \textless 10 & \textless 5 \\[0.75em]
\parbox[c]{2cm}{Fine, \mbox{(Thin)}} & 1 to \textless 2 & 10 to \textless 20 & 5 to \textless 10 \\[0.75em]
{Medium, \mbox{(Medium)}} & 2 to \textless 5 & 20 to \textless 50 & 10 to \textless 20 \\[0.75em]
{Coarse, \mbox{(Thick)}} & 5 to \textless 10 & 50 to \textless 100 & \textless 20 to \textless 50 \\[0.75em]
{Very Coarse, \mbox{(Very Thick)}} & $\geq$ 10 & 100 to \textless 500 & $\geq$ 50 \\[0.75em]
{Extremely Coarse} & \textendash & $\geq$ 500 & \textendash \\[0.75em]
\hline
\end{tabular}
\end{table}
\end{document}