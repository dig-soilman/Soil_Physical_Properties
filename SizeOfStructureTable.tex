% Donald G. McGahan 2022 v. 01
\documentclass{memoir}
\usepackage[utf8]{inputenc}
\usepackage{multirow} % for multi-row in tables
\begin{document}
\subsection{Size of Structure}
\label{sizeofstructure}
    
Size is conveyed in such a way that the peds narrowest dimension  determines the size class for each type. Size classes for granular, angular blocky, sub-angular blocky, lenticular, wedge, prismatic, and columnar structure types are \emph{very fine}, \emph{fine}, \emph{medium}, \emph{coarse}, and \emph{very coarse}. For the platy structure type thin is substituted for fine and thick is substituted for coarse size class names.

% dig-soilman: revised but waiting to see if this structure size class table translates to html via pandoc well.

\begin{table}[!htbp]
\label{tab:astructuresizeclass}
\centering
\caption{Structure size classes.}
\begin{tabular}{l c c c }
\hline \\[0.15em]
Size Class & \multicolumn{3}{c}{Criteria: Structural unit size (mm)}\\[0.75em] 
\cline{2-4} \\
& \parbox[c]{2cm}{Granular, \mbox{(Diameter)} Platy \mbox{(Thickness)}} & \parbox[c]{2cm}{Columnar, Prismatic, Wedge, \mbox{(Diameter)}} & \parbox[c]{2cm}{Angular \& Subangular Blocky and Lenticular, \mbox{(Diameter)}} \\[0.75em]
 \hline\\[0.75em]
{Very Fine, \mbox{(Very Thin)}} & \textless\,1 & \textless\,10 & \textless\,5 \\[0.75em]
\parbox[c]{2cm}{Fine, \mbox{(Thin)}} & 1\,to\,\textless\,2 & 10\,to\,\textless\,20 & 5\,to\,\textless\,10 \\[0.75em]
{Medium, \mbox{(Medium)}} & 2\,to\,\textless\,5 & 20\,to\,\textless\,50 & 10\,to\,\textless\,20 \\[0.75em]
{Coarse, \mbox{(Thick)}} & 5\,to\,\textless\,10 & 50\,to\,\textless\,100 & \textless\,20\,to\,\textless\,50 \\[0.75em]
{Very Coarse, \mbox{(Very Thick)}} & $\geq$\,10 & 100\,to\,\textless\,500 & $\geq$\,50 \\[0.75em]
{Extremely Coarse} & \textendash & $\geq$\,500 & \textendash \\[0.75em]
\hline
\end{tabular}
\end{table}
\end{document}